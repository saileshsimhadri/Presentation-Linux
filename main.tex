\documentclass{beamer}


% Get rid of the navigation bar at the bottom
\beamertemplatenavigationsymbolsempty

\usetheme{Frankfurt}

\title{UPE Linux Introduction}

\author{Drew Monroe, Sailesh Simhadri, Joseph Sweeney, Soumya Kundu}

\institute[Upsilon Pi Epsilon - University of Connecticut]

\date{\today}

\subject{Linux}

% Let's get started
\begin{document}

\begin{frame}
  \titlepage
\end{frame}

\begin{frame}{Outline}
  \tableofcontents
  % You might wish to add the option [pausesections]
\end{frame}

%% Leaving this slide in as a reference, it will be removed later

% Section and subsections will appear in the presentation overview
% and table of contents.
\section{First Main Section}

\subsection{First Subsection}

\begin{frame}{First Slide Title}{Optional Subtitle}
  \begin{itemize}
  \item {
    My first point.
  }
  \item {
    My second point.
  }
  \end{itemize}
\end{frame}

\section{Installing Programs}

\subsection{Package Managers}

\begin{frame}{Package Managers}
    \begin{itemize}
        \item An ``app store'' for Linux
        \item Automate the process of managing programs
        \item Eliminate the need for manual installation
        \item Different distributions use different package managers (Ubuntu's is called apt)
    \end{itemize}
\end{frame}

\begin{frame}{Package Managers}
  \begin{itemize}
      \item The package manager can do many different things: install, update, remove
      \item These packages that get installed come from repositories (servers that store the packages)
      \item These repositories can become outdated and need to be updated
      \item sudo apt update
  \end{itemize}
\end{frame}

\begin{frame}{Package Managers - Usage}
    \begin{itemize}
        \item Installation install command
            \begin{itemize}
                \item sudo apt install *package name*
            \end{itemize}
        \item Upgrade command
            \begin{itemize}
                \item sudo apt upgrade
            \end{itemize}
        \item Remove command
            \begin{itemize}
                \item sudo apt remove *package name*
            \end{itemize}
    \end{itemize}
\end{frame}

% Placing a * after \section means it will not show in the
% outline or table of contents.
\section{Miscellaneous}

\subsection{Where do I find files?}

\begin{frame}{Where to find files}
    \begin{itemize}
        \item \url{bin} - Binary files
        \item \url{boot} - Files required to boot the system
        \item \url{dev} - Device files, disk partitions, standard in and out
        \item \url{etc} - System configuration files
        \item \url{home} - Users home directories
        \item \url{lib} - Libraries and modules, including kernel modules
        \item \url{lib64} - 64 bit versions of modules
        \item \url{opt} -Traditionally used for third party add-ons
        \item \url{proc} - Runtime system information, ``files'' about the system
        \item \url{root} - The home directory for root
        \item \url{sbin} - Similar to \url{bin}, but root executables
        \item \url{tmp} - Temporary files
        \item \url{usr} - User programs, documentation, static data; originally like \url{home}
        \item \url{var} - Variable data, logs, application data
    \end{itemize}
\end{frame}



% All of the following is optional and typically not needed. 
\appendix
\section<presentation>*{\appendixname}
\subsection<presentation>*{For Further Reading}

\begin{frame}[allowframebreaks]
  \frametitle<presentation>{For Further Reading}
    
  \begin{thebibliography}{10}
    
  \beamertemplatebookbibitems
  % Start with overview books.

  \bibitem{Author1990}
    A.~Author.
    \newblock {\em Handbook of Everything}.
    \newblock Some Press, 1990.
 
    
  \beamertemplatearticlebibitems
  % Followed by interesting articles. Keep the list short. 

  \bibitem{Someone2000}
    S.~Someone.
    \newblock On this and that.
    \newblock {\em Journal of This and That}, 2(1):50--100,
    2000.
  \end{thebibliography}
\end{frame}

\end{document}
